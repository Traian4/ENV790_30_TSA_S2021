% Options for packages loaded elsewhere
\PassOptionsToPackage{unicode}{hyperref}
\PassOptionsToPackage{hyphens}{url}
%
\documentclass[
]{article}
\usepackage{amsmath,amssymb}
\usepackage{lmodern}
\usepackage{ifxetex,ifluatex}
\ifnum 0\ifxetex 1\fi\ifluatex 1\fi=0 % if pdftex
  \usepackage[T1]{fontenc}
  \usepackage[utf8]{inputenc}
  \usepackage{textcomp} % provide euro and other symbols
\else % if luatex or xetex
  \usepackage{unicode-math}
  \defaultfontfeatures{Scale=MatchLowercase}
  \defaultfontfeatures[\rmfamily]{Ligatures=TeX,Scale=1}
\fi
% Use upquote if available, for straight quotes in verbatim environments
\IfFileExists{upquote.sty}{\usepackage{upquote}}{}
\IfFileExists{microtype.sty}{% use microtype if available
  \usepackage[]{microtype}
  \UseMicrotypeSet[protrusion]{basicmath} % disable protrusion for tt fonts
}{}
\makeatletter
\@ifundefined{KOMAClassName}{% if non-KOMA class
  \IfFileExists{parskip.sty}{%
    \usepackage{parskip}
  }{% else
    \setlength{\parindent}{0pt}
    \setlength{\parskip}{6pt plus 2pt minus 1pt}}
}{% if KOMA class
  \KOMAoptions{parskip=half}}
\makeatother
\usepackage{xcolor}
\IfFileExists{xurl.sty}{\usepackage{xurl}}{} % add URL line breaks if available
\IfFileExists{bookmark.sty}{\usepackage{bookmark}}{\usepackage{hyperref}}
\hypersetup{
  pdftitle={ENV 790.30 - Time Series Analysis for Energy Data \textbar{} Spring 2021},
  pdfauthor={Traian Nirca},
  hidelinks,
  pdfcreator={LaTeX via pandoc}}
\urlstyle{same} % disable monospaced font for URLs
\usepackage[margin=2.54cm]{geometry}
\usepackage{color}
\usepackage{fancyvrb}
\newcommand{\VerbBar}{|}
\newcommand{\VERB}{\Verb[commandchars=\\\{\}]}
\DefineVerbatimEnvironment{Highlighting}{Verbatim}{commandchars=\\\{\}}
% Add ',fontsize=\small' for more characters per line
\usepackage{framed}
\definecolor{shadecolor}{RGB}{248,248,248}
\newenvironment{Shaded}{\begin{snugshade}}{\end{snugshade}}
\newcommand{\AlertTok}[1]{\textcolor[rgb]{0.94,0.16,0.16}{#1}}
\newcommand{\AnnotationTok}[1]{\textcolor[rgb]{0.56,0.35,0.01}{\textbf{\textit{#1}}}}
\newcommand{\AttributeTok}[1]{\textcolor[rgb]{0.77,0.63,0.00}{#1}}
\newcommand{\BaseNTok}[1]{\textcolor[rgb]{0.00,0.00,0.81}{#1}}
\newcommand{\BuiltInTok}[1]{#1}
\newcommand{\CharTok}[1]{\textcolor[rgb]{0.31,0.60,0.02}{#1}}
\newcommand{\CommentTok}[1]{\textcolor[rgb]{0.56,0.35,0.01}{\textit{#1}}}
\newcommand{\CommentVarTok}[1]{\textcolor[rgb]{0.56,0.35,0.01}{\textbf{\textit{#1}}}}
\newcommand{\ConstantTok}[1]{\textcolor[rgb]{0.00,0.00,0.00}{#1}}
\newcommand{\ControlFlowTok}[1]{\textcolor[rgb]{0.13,0.29,0.53}{\textbf{#1}}}
\newcommand{\DataTypeTok}[1]{\textcolor[rgb]{0.13,0.29,0.53}{#1}}
\newcommand{\DecValTok}[1]{\textcolor[rgb]{0.00,0.00,0.81}{#1}}
\newcommand{\DocumentationTok}[1]{\textcolor[rgb]{0.56,0.35,0.01}{\textbf{\textit{#1}}}}
\newcommand{\ErrorTok}[1]{\textcolor[rgb]{0.64,0.00,0.00}{\textbf{#1}}}
\newcommand{\ExtensionTok}[1]{#1}
\newcommand{\FloatTok}[1]{\textcolor[rgb]{0.00,0.00,0.81}{#1}}
\newcommand{\FunctionTok}[1]{\textcolor[rgb]{0.00,0.00,0.00}{#1}}
\newcommand{\ImportTok}[1]{#1}
\newcommand{\InformationTok}[1]{\textcolor[rgb]{0.56,0.35,0.01}{\textbf{\textit{#1}}}}
\newcommand{\KeywordTok}[1]{\textcolor[rgb]{0.13,0.29,0.53}{\textbf{#1}}}
\newcommand{\NormalTok}[1]{#1}
\newcommand{\OperatorTok}[1]{\textcolor[rgb]{0.81,0.36,0.00}{\textbf{#1}}}
\newcommand{\OtherTok}[1]{\textcolor[rgb]{0.56,0.35,0.01}{#1}}
\newcommand{\PreprocessorTok}[1]{\textcolor[rgb]{0.56,0.35,0.01}{\textit{#1}}}
\newcommand{\RegionMarkerTok}[1]{#1}
\newcommand{\SpecialCharTok}[1]{\textcolor[rgb]{0.00,0.00,0.00}{#1}}
\newcommand{\SpecialStringTok}[1]{\textcolor[rgb]{0.31,0.60,0.02}{#1}}
\newcommand{\StringTok}[1]{\textcolor[rgb]{0.31,0.60,0.02}{#1}}
\newcommand{\VariableTok}[1]{\textcolor[rgb]{0.00,0.00,0.00}{#1}}
\newcommand{\VerbatimStringTok}[1]{\textcolor[rgb]{0.31,0.60,0.02}{#1}}
\newcommand{\WarningTok}[1]{\textcolor[rgb]{0.56,0.35,0.01}{\textbf{\textit{#1}}}}
\usepackage{graphicx}
\makeatletter
\def\maxwidth{\ifdim\Gin@nat@width>\linewidth\linewidth\else\Gin@nat@width\fi}
\def\maxheight{\ifdim\Gin@nat@height>\textheight\textheight\else\Gin@nat@height\fi}
\makeatother
% Scale images if necessary, so that they will not overflow the page
% margins by default, and it is still possible to overwrite the defaults
% using explicit options in \includegraphics[width, height, ...]{}
\setkeys{Gin}{width=\maxwidth,height=\maxheight,keepaspectratio}
% Set default figure placement to htbp
\makeatletter
\def\fps@figure{htbp}
\makeatother
\setlength{\emergencystretch}{3em} % prevent overfull lines
\providecommand{\tightlist}{%
  \setlength{\itemsep}{0pt}\setlength{\parskip}{0pt}}
\setcounter{secnumdepth}{-\maxdimen} % remove section numbering
\usepackage{enumerate}
\usepackage{enumitem}
\usepackage[utf8]{inputenc}
\DeclareUnicodeCharacter{00A0}{ }
\usepackage{booktabs}
\usepackage{longtable}
\usepackage{array}
\usepackage{multirow}
\usepackage{wrapfig}
\usepackage{float}
\usepackage{colortbl}
\usepackage{pdflscape}
\usepackage{tabu}
\usepackage{threeparttable}
\usepackage{threeparttablex}
\usepackage[normalem]{ulem}
\usepackage{makecell}
\usepackage{xcolor}
\ifluatex
  \usepackage{selnolig}  % disable illegal ligatures
\fi

\title{ENV 790.30 - Time Series Analysis for Energy Data \textbar{}
Spring 2021}
\usepackage{etoolbox}
\makeatletter
\providecommand{\subtitle}[1]{% add subtitle to \maketitle
  \apptocmd{\@title}{\par {\large #1 \par}}{}{}
}
\makeatother
\subtitle{Assignment 7 - Due date 04/07/21}
\author{Traian Nirca}
\date{}

\begin{document}
\maketitle

\textbackslash begin\{document\}

\hypertarget{directions}{%
\subsection{Directions}\label{directions}}

You should open the .rmd file corresponding to this assignment on
RStudio. The file is available on our class repository on Github. And to
do so you will need to fork our repository and link it to your RStudio.

Once you have the project open the first thing you will do is change
``Student Name'' on line 3 with your name. Then you will start working
through the assignment by \textbf{creating code and output} that answer
each question. Be sure to use this assignment document. Your report
should contain the answer to each question and any plots/tables you
obtained (when applicable).

When you have completed the assignment, \textbf{Knit} the text and code
into a single PDF file. Rename the pdf file such that it includes your
first and last name (e.g., ``LuanaLima\_TSA\_A07\_Sp21.Rmd''). Submit
this pdf using Sakai.

\hypertarget{set-up}{%
\subsection{Set up}\label{set-up}}

Some packages needed for this assignment:
\texttt{forecast},\texttt{tseries},\texttt{smooth}. Do not forget to
load them before running your script, since they are NOT default
packages.

\begin{Shaded}
\begin{Highlighting}[]
\CommentTok{\#Load/install required package here}
\FunctionTok{library}\NormalTok{(forecast)}
\end{Highlighting}
\end{Shaded}

\begin{verbatim}
## Registered S3 method overwritten by 'quantmod':
##   method            from
##   as.zoo.data.frame zoo
\end{verbatim}

\begin{Shaded}
\begin{Highlighting}[]
\FunctionTok{library}\NormalTok{(tseries)}
\FunctionTok{library}\NormalTok{(smooth)}
\end{Highlighting}
\end{Shaded}

\begin{verbatim}
## Loading required package: greybox
\end{verbatim}

\begin{verbatim}
## Package "greybox", v0.6.8 loaded.
\end{verbatim}

\begin{verbatim}
## This is package "smooth", v3.1.0
\end{verbatim}

\begin{Shaded}
\begin{Highlighting}[]
\FunctionTok{library}\NormalTok{(ggplot2)}
\FunctionTok{library}\NormalTok{(lubridate)}
\end{Highlighting}
\end{Shaded}

\begin{verbatim}
## 
## Attaching package: 'lubridate'
\end{verbatim}

\begin{verbatim}
## The following object is masked from 'package:greybox':
## 
##     hm
\end{verbatim}

\begin{verbatim}
## The following objects are masked from 'package:base':
## 
##     date, intersect, setdiff, union
\end{verbatim}

\begin{Shaded}
\begin{Highlighting}[]
\FunctionTok{library}\NormalTok{(kableExtra)}
\end{Highlighting}
\end{Shaded}

\hypertarget{importing-and-processing-the-data-set}{%
\subsection{Importing and processing the data
set}\label{importing-and-processing-the-data-set}}

Consider the data from the file ``inflowtimeseries.txt''. The data
corresponds to the monthly inflow in \(m^{3}/s\) for some hydro power
plants in Brazil. You will only use the last column of the data set
which represents one hydro plant in the Amazon river basin. The data
span the period from January 1931 to August 2011 and is provided by the
Brazilian ISO.

For all parts of the assignment prepare the data set such that the model
consider only the data from January 2000 up to December 2009. Leave the
year 2010 of data (January 2010 to December 2010) for the out-of-sample
analysis. Do \textbf{NOT} use data fro 2010 and 2011 for model fitting.
You will only use it to compute forecast accuracy of your model.

\hypertarget{part-i-preparing-the-data-sets}{%
\subsection{Part I: Preparing the data
sets}\label{part-i-preparing-the-data-sets}}

\hypertarget{q1}{%
\subsubsection{Q1}\label{q1}}

Read the file into a data frame. Prepare your time series data vector
such that observations start in January 2000 and end in December 2009.
Make you sure you specify the \textbf{start=} and \textbf{frequency=}
arguments. Plot the time series over time, ACF and PACF.

\begin{Shaded}
\begin{Highlighting}[]
\NormalTok{raw\_data }\OtherTok{\textless{}{-}} \FunctionTok{read.table}\NormalTok{(}\AttributeTok{file=}\StringTok{"../Data/inflowtimeseries.txt"}\NormalTok{, }\AttributeTok{header=}\ConstantTok{FALSE}\NormalTok{, }\AttributeTok{skip =} \DecValTok{828}\NormalTok{, }\AttributeTok{nrows =} \DecValTok{120}\NormalTok{)}

\NormalTok{work\_data }\OtherTok{\textless{}{-}} \FunctionTok{data.frame}\NormalTok{(}\StringTok{"Monthly Inflow"}\OtherTok{=}\NormalTok{raw\_data[,}\DecValTok{17}\NormalTok{])}

\NormalTok{time\_series }\OtherTok{\textless{}{-}} \FunctionTok{ts}\NormalTok{(work\_data, }\AttributeTok{frequency =} \DecValTok{12}\NormalTok{, }\AttributeTok{start =} \FunctionTok{c}\NormalTok{(}\DecValTok{2000}\NormalTok{,}\DecValTok{1}\NormalTok{))}
\FunctionTok{head}\NormalTok{(time\_series)}
\end{Highlighting}
\end{Shaded}

\begin{verbatim}
##        Jan   Feb   Mar   Apr   May   Jun
## 2000 22107 29247 36651 34089 25821 18007
\end{verbatim}

\begin{Shaded}
\begin{Highlighting}[]
\NormalTok{lastyear\_raw }\OtherTok{\textless{}{-}} \FunctionTok{read.table}\NormalTok{(}\AttributeTok{file=}\StringTok{"../Data/inflowtimeseries.txt"}\NormalTok{, }\AttributeTok{header=}\ConstantTok{FALSE}\NormalTok{, }\AttributeTok{skip =} \DecValTok{828}\NormalTok{, }\AttributeTok{nrows =} \DecValTok{132}\NormalTok{)}
\NormalTok{lastyear\_work }\OtherTok{\textless{}{-}} \FunctionTok{data.frame}\NormalTok{(}\StringTok{"Monthly Inflow"}\OtherTok{=}\NormalTok{lastyear\_raw[,}\DecValTok{17}\NormalTok{])}
\NormalTok{lastyear\_ts }\OtherTok{\textless{}{-}} \FunctionTok{ts}\NormalTok{(lastyear\_work, }\AttributeTok{frequency =} \DecValTok{12}\NormalTok{, }\AttributeTok{start =} \FunctionTok{c}\NormalTok{(}\DecValTok{2000}\NormalTok{,}\DecValTok{1}\NormalTok{))}

\NormalTok{the\_date }\OtherTok{\textless{}{-}} \FunctionTok{paste}\NormalTok{(raw\_data[,}\DecValTok{1}\NormalTok{], raw\_data[,}\DecValTok{2}\NormalTok{])}
\NormalTok{my\_date }\OtherTok{\textless{}{-}} \FunctionTok{my}\NormalTok{(the\_date)}
\FunctionTok{head}\NormalTok{(my\_date)}
\end{Highlighting}
\end{Shaded}

\begin{verbatim}
## [1] "2000-01-01" "2000-02-01" "2000-03-01" "2000-04-01" "2000-05-01"
## [6] "2000-06-01"
\end{verbatim}

\begin{Shaded}
\begin{Highlighting}[]
\NormalTok{the\_date2 }\OtherTok{\textless{}{-}} \FunctionTok{paste}\NormalTok{(lastyear\_raw[,}\DecValTok{1}\NormalTok{], lastyear\_raw[,}\DecValTok{2}\NormalTok{])}
\NormalTok{my\_date2 }\OtherTok{\textless{}{-}} \FunctionTok{my}\NormalTok{(the\_date2)}


\FunctionTok{plot}\NormalTok{(time\_series, }\AttributeTok{ylab=}\StringTok{"Monthly Inflow (m\^{}3/s)"}\NormalTok{)}
\end{Highlighting}
\end{Shaded}

\includegraphics{TSA_A7_Sp21_files/figure-latex/unnamed-chunk-2-1.pdf}

\begin{Shaded}
\begin{Highlighting}[]
\FunctionTok{par}\NormalTok{(}\AttributeTok{mfrow=}\FunctionTok{c}\NormalTok{(}\DecValTok{1}\NormalTok{,}\DecValTok{2}\NormalTok{))}
\FunctionTok{Acf}\NormalTok{(time\_series, }\AttributeTok{lag.max =} \DecValTok{40}\NormalTok{)}
\FunctionTok{Pacf}\NormalTok{(time\_series, }\AttributeTok{lag.max =} \DecValTok{40}\NormalTok{)}
\end{Highlighting}
\end{Shaded}

\includegraphics{TSA_A7_Sp21_files/figure-latex/unnamed-chunk-2-2.pdf}

\hypertarget{q2}{%
\subsubsection{Q2}\label{q2}}

Using the \(decompose()\) or \(stl()\) and the \(seasadj()\) functions
create a series without the seasonal component, i.e., a deseasonalized
inflow series. Plot the deseasonalized series and original series
together using ggplot, make sure your plot includes a legend. Plot ACF
and PACF for the deaseasonalized series. Compare with the plots obtained
in Q1.

\begin{Shaded}
\begin{Highlighting}[]
\NormalTok{decomposed\_series }\OtherTok{\textless{}{-}} \FunctionTok{decompose}\NormalTok{(time\_series,}\StringTok{"additive"}\NormalTok{)}
\FunctionTok{plot}\NormalTok{(decomposed\_series)}
\end{Highlighting}
\end{Shaded}

\includegraphics{TSA_A7_Sp21_files/figure-latex/unnamed-chunk-3-1.pdf}

\begin{Shaded}
\begin{Highlighting}[]
\NormalTok{seasadj\_series }\OtherTok{\textless{}{-}} \FunctionTok{seasadj}\NormalTok{(decomposed\_series)}
\FunctionTok{plot}\NormalTok{(seasadj\_series)}
\end{Highlighting}
\end{Shaded}

\includegraphics{TSA_A7_Sp21_files/figure-latex/unnamed-chunk-3-2.pdf}

\begin{Shaded}
\begin{Highlighting}[]
\NormalTok{decomposed\_lastyear }\OtherTok{\textless{}{-}} \FunctionTok{decompose}\NormalTok{(lastyear\_ts,}\StringTok{"additive"}\NormalTok{)}
\NormalTok{seasadj\_lastyear }\OtherTok{\textless{}{-}} \FunctionTok{seasadj}\NormalTok{(decomposed\_lastyear)}

\FunctionTok{ggplot}\NormalTok{(time\_series[,}\DecValTok{1}\NormalTok{], }\FunctionTok{aes}\NormalTok{(}\AttributeTok{x=}\NormalTok{my\_date)) }\SpecialCharTok{+}
  \FunctionTok{geom\_line}\NormalTok{(}\FunctionTok{aes}\NormalTok{(}\AttributeTok{y=}\NormalTok{time\_series[,}\DecValTok{1}\NormalTok{], }\AttributeTok{col=}\StringTok{"Original Series"}\NormalTok{)) }\SpecialCharTok{+}
  \FunctionTok{geom\_line}\NormalTok{(}\FunctionTok{aes}\NormalTok{(}\AttributeTok{y=}\NormalTok{seasadj\_series, }\AttributeTok{col=}\StringTok{"Deseasoned Series"}\NormalTok{))}\SpecialCharTok{+}
  \FunctionTok{ylab}\NormalTok{(}\StringTok{"Monthly Inflow (m\^{}3/s)"}\NormalTok{) }\SpecialCharTok{+}
  \FunctionTok{xlab}\NormalTok{(}\StringTok{"Date"}\NormalTok{)}
\end{Highlighting}
\end{Shaded}

\begin{verbatim}
## Don't know how to automatically pick scale for object of type ts. Defaulting to continuous.
\end{verbatim}

\includegraphics{TSA_A7_Sp21_files/figure-latex/unnamed-chunk-3-3.pdf}

\begin{Shaded}
\begin{Highlighting}[]
\FunctionTok{par}\NormalTok{(}\AttributeTok{mfrow=}\FunctionTok{c}\NormalTok{(}\DecValTok{1}\NormalTok{,}\DecValTok{2}\NormalTok{))}
\FunctionTok{Acf}\NormalTok{(seasadj\_series, }\AttributeTok{lag.max =} \DecValTok{40}\NormalTok{)}
\FunctionTok{Pacf}\NormalTok{(seasadj\_series, }\AttributeTok{lag.max =} \DecValTok{40}\NormalTok{)}
\end{Highlighting}
\end{Shaded}

\includegraphics{TSA_A7_Sp21_files/figure-latex/unnamed-chunk-3-4.pdf}
\textgreater Answer: We certainly see a difference between these plots
and those from Q1, as there is an (obvious) absence of the seasonal
trend. The ACF of the seasonaly adjusted series does maintain a degree
of seasonality.

\hypertarget{part-ii-forecasting-with-arima-models-and-its-variations}{%
\subsection{Part II: Forecasting with ARIMA models and its
variations}\label{part-ii-forecasting-with-arima-models-and-its-variations}}

\hypertarget{q3}{%
\subsubsection{Q3}\label{q3}}

Fit a non-seasonal ARIMA\((p,d,q)\) model using the auto.arima()
function to the non-seasonal data. Forecast 12 months ahead of time
using the \(forecast()\) function. Plot your forecasting results and
further include on the plot the last year of non-seasonal data to
compare with forecasted values (similar to the plot on the lesson file
for M10).

\begin{Shaded}
\begin{Highlighting}[]
\NormalTok{ARIMA\_fit }\OtherTok{\textless{}{-}} \FunctionTok{auto.arima}\NormalTok{(seasadj\_series)}
\FunctionTok{checkresiduals}\NormalTok{(ARIMA\_fit)}
\end{Highlighting}
\end{Shaded}

\includegraphics{TSA_A7_Sp21_files/figure-latex/unnamed-chunk-4-1.pdf}

\begin{verbatim}
## 
##  Ljung-Box test
## 
## data:  Residuals from ARIMA(0,1,0)
## Q* = 24.208, df = 24, p-value = 0.4498
## 
## Model df: 0.   Total lags used: 24
\end{verbatim}

\begin{Shaded}
\begin{Highlighting}[]
\NormalTok{ARIMA\_forecast }\OtherTok{\textless{}{-}} \FunctionTok{forecast}\NormalTok{(ARIMA\_fit,}\AttributeTok{h=}\DecValTok{12}\NormalTok{)}

\FunctionTok{plot}\NormalTok{(ARIMA\_forecast)}
\FunctionTok{lines}\NormalTok{(seasadj\_lastyear,}\AttributeTok{col=}\StringTok{"green"}\NormalTok{)}
\end{Highlighting}
\end{Shaded}

\includegraphics{TSA_A7_Sp21_files/figure-latex/unnamed-chunk-4-2.pdf}

The result seems very strange, it looks like a mean forecast. This is
possibly due to the lack of seasonality.

\hypertarget{q4}{%
\subsubsection{Q4}\label{q4}}

Put the seasonality back on your forecasted values and compare with the
original seasonal data values. \(Hint:\) One way to do it is by summing
the last year of the seasonal component from your decompose object to
the forecasted series.

\begin{Shaded}
\begin{Highlighting}[]
\NormalTok{Temp }\OtherTok{\textless{}{-}}\NormalTok{ decomposed\_series}\SpecialCharTok{$}\NormalTok{seasonal}
\NormalTok{Temp1 }\OtherTok{\textless{}{-}} \FunctionTok{snaive}\NormalTok{(Temp, }\DecValTok{12}\NormalTok{, }\AttributeTok{holdout=}\ConstantTok{FALSE}\NormalTok{)}
\NormalTok{Temp3 }\OtherTok{\textless{}{-}}\NormalTok{ ARIMA\_forecast}
\NormalTok{Temp3}\SpecialCharTok{$}\NormalTok{mean }\OtherTok{\textless{}{-}}\NormalTok{ Temp3[[}\StringTok{"mean"}\NormalTok{]] }\SpecialCharTok{+}\NormalTok{ Temp1[[}\StringTok{"mean"}\NormalTok{]]}

\FunctionTok{plot}\NormalTok{(Temp3)}
\end{Highlighting}
\end{Shaded}

\includegraphics{TSA_A7_Sp21_files/figure-latex/unnamed-chunk-5-1.pdf}
Took me a while, but this still does not look right.

\hypertarget{q5}{%
\subsubsection{Q5}\label{q5}}

Repeat Q3 for the original data, but now fit a seasonal
ARIMA\((p,d,q)x(P,D,Q)_ {12}\) also using the auto.arima().

\begin{Shaded}
\begin{Highlighting}[]
\NormalTok{SARIMA\_fit }\OtherTok{\textless{}{-}} \FunctionTok{auto.arima}\NormalTok{(time\_series)}
\FunctionTok{checkresiduals}\NormalTok{(SARIMA\_fit)}
\end{Highlighting}
\end{Shaded}

\includegraphics{TSA_A7_Sp21_files/figure-latex/unnamed-chunk-6-1.pdf}

\begin{verbatim}
## 
##  Ljung-Box test
## 
## data:  Residuals from ARIMA(1,0,0)(0,1,1)[12] with drift
## Q* = 19.881, df = 21, p-value = 0.5288
## 
## Model df: 3.   Total lags used: 24
\end{verbatim}

\begin{Shaded}
\begin{Highlighting}[]
\NormalTok{SARIMA\_forecast }\OtherTok{\textless{}{-}} \FunctionTok{forecast}\NormalTok{(SARIMA\_fit,}\AttributeTok{h=}\DecValTok{12}\NormalTok{)}

\FunctionTok{plot}\NormalTok{(SARIMA\_forecast)}
\FunctionTok{lines}\NormalTok{(lastyear\_ts,}\AttributeTok{col=}\StringTok{"green"}\NormalTok{)}
\end{Highlighting}
\end{Shaded}

\includegraphics{TSA_A7_Sp21_files/figure-latex/unnamed-chunk-6-2.pdf}

\hypertarget{q6}{%
\subsubsection{Q6}\label{q6}}

Compare the plots from Q4 and Q5 using the autoplot() function.

\begin{Shaded}
\begin{Highlighting}[]
\FunctionTok{par}\NormalTok{(}\AttributeTok{mfrow=}\FunctionTok{c}\NormalTok{(}\DecValTok{1}\NormalTok{,}\DecValTok{2}\NormalTok{))}
\FunctionTok{autoplot}\NormalTok{(Temp3)}
\end{Highlighting}
\end{Shaded}

\includegraphics{TSA_A7_Sp21_files/figure-latex/unnamed-chunk-7-1.pdf}

\begin{Shaded}
\begin{Highlighting}[]
\FunctionTok{autoplot}\NormalTok{(SARIMA\_forecast)}
\end{Highlighting}
\end{Shaded}

\includegraphics{TSA_A7_Sp21_files/figure-latex/unnamed-chunk-7-2.pdf}
The plot from Q4 is clearly wrong, while the one from Q5 looks much more
fitting. There seems to be a level of discontinuity however.

\hypertarget{part-iii-forecasting-with-other-models}{%
\subsection{Part III: Forecasting with Other
Models}\label{part-iii-forecasting-with-other-models}}

\hypertarget{q7}{%
\subsubsection{Q7}\label{q7}}

Fit an exponential smooth model to the original time series using the
function \(es()\) from package \texttt{smooth}. Note that this function
automatically do the forecast. Do not forget to set the arguments:
silent=FALSE and holdout=FALSE, so that the plot is produced and the
forecast is for the year of 2010.

\begin{Shaded}
\begin{Highlighting}[]
\NormalTok{Exp\_smooth }\OtherTok{\textless{}{-}} \FunctionTok{es}\NormalTok{(time\_series, }\AttributeTok{h=}\DecValTok{12}\NormalTok{,}\AttributeTok{holdout=}\ConstantTok{FALSE}\NormalTok{,}\AttributeTok{silent=}\ConstantTok{FALSE}\NormalTok{)}
\end{Highlighting}
\end{Shaded}

\begin{verbatim}
## Forming the pool of models based on... ANN, ANA, MNM, MAM, Estimation progress:    100%... Done!
\end{verbatim}

\includegraphics{TSA_A7_Sp21_files/figure-latex/unnamed-chunk-8-1.pdf}

\begin{Shaded}
\begin{Highlighting}[]
\FunctionTok{plot}\NormalTok{(Exp\_smooth)}
\end{Highlighting}
\end{Shaded}

\includegraphics{TSA_A7_Sp21_files/figure-latex/unnamed-chunk-8-2.pdf}
\includegraphics{TSA_A7_Sp21_files/figure-latex/unnamed-chunk-8-3.pdf}
\includegraphics{TSA_A7_Sp21_files/figure-latex/unnamed-chunk-8-4.pdf}
\includegraphics{TSA_A7_Sp21_files/figure-latex/unnamed-chunk-8-5.pdf}

\begin{Shaded}
\begin{Highlighting}[]
\FunctionTok{checkresiduals}\NormalTok{(Exp\_smooth)}
\end{Highlighting}
\end{Shaded}

\begin{verbatim}
## Warning in modeldf.default(object): Could not find appropriate degrees of
## freedom for this model.
\end{verbatim}

\includegraphics{TSA_A7_Sp21_files/figure-latex/unnamed-chunk-8-6.pdf}

\hypertarget{q8}{%
\subsubsection{Q8}\label{q8}}

Fit a state space model to the original time series using the function
\(StructTS()\) from package \texttt{stats}. Which one of the tree model
we learned should you try: ``local'', ``trend'', or ``BSM''. Why? Play
with argument \texttt{fixed} a bit to try to understand how the
different variances can affect the model. If you can't seem to find a
variance that leads to a good fit here is a hint: try
\(fixed=c(0.1, 0.001, NA, NA)\). Since \(StructTS()\) fits a state space
model to the data, you need to use \(forecast()\) to generate the
forecasts. Like you do for the ARIMA fit.

\begin{Shaded}
\begin{Highlighting}[]
\NormalTok{SS\_model }\OtherTok{\textless{}{-}} \FunctionTok{StructTS}\NormalTok{(time\_series,}
                    \AttributeTok{type=}\StringTok{"BSM"}\NormalTok{,}\AttributeTok{fixed=}\FunctionTok{c}\NormalTok{(}\FloatTok{0.1}\NormalTok{,}\FloatTok{0.001}\NormalTok{,}\ConstantTok{NA}\NormalTok{,}\ConstantTok{NA}\NormalTok{)) }
\FunctionTok{checkresiduals}\NormalTok{(SS\_model)}
\end{Highlighting}
\end{Shaded}

\begin{verbatim}
## Warning in modeldf.default(object): Could not find appropriate degrees of
## freedom for this model.
\end{verbatim}

\includegraphics{TSA_A7_Sp21_files/figure-latex/unnamed-chunk-9-1.pdf}

\begin{Shaded}
\begin{Highlighting}[]
\NormalTok{SS\_for }\OtherTok{\textless{}{-}} \FunctionTok{forecast}\NormalTok{(SS\_model,}\AttributeTok{h=}\DecValTok{12}\NormalTok{)}
\FunctionTok{plot}\NormalTok{(SS\_for)}
\end{Highlighting}
\end{Shaded}

\includegraphics{TSA_A7_Sp21_files/figure-latex/unnamed-chunk-9-2.pdf}
We use BSM because it is a seasonal series, with a frequency
\textgreater{} 1

\hypertarget{part-iv-checking-forecast-accuracy}{%
\subsection{Part IV: Checking Forecast
Accuracy}\label{part-iv-checking-forecast-accuracy}}

\hypertarget{q9}{%
\subsubsection{Q9}\label{q9}}

Make one plot with the complete original seasonal historical data (Jan
2000 to Dec 2010). Now add the forecasts from each of the developed
models in parts Q4, Q5, Q7 and Q8. You can do it using the autoplot()
combined with autolayer(). If everything is correct in terms of time
line, the forecasted lines should appear only in the final year. If you
decide to use ggplot() you will need to create a data frame with all the
series will need to plot. Remember to use a different color for each
model and add a legend in the end to tell which forecast lines
corresponds to each model.

\begin{Shaded}
\begin{Highlighting}[]
\NormalTok{Inflow }\OtherTok{\textless{}{-}} \FunctionTok{cbind}\NormalTok{(}\StringTok{"Original Data"} \OtherTok{=}\NormalTok{lastyear\_ts, }\StringTok{"SARIMA"}\OtherTok{=}\NormalTok{SARIMA\_forecast}\SpecialCharTok{$}\NormalTok{mean, }\StringTok{"ARIMA"}\OtherTok{=}\NormalTok{Temp3}\SpecialCharTok{$}\NormalTok{mean, }\StringTok{"Exponential"} \OtherTok{=}\NormalTok{ Exp\_smooth}\SpecialCharTok{$}\NormalTok{forecast, }\StringTok{"State Space"}\OtherTok{=}\NormalTok{SS\_for}\SpecialCharTok{$}\NormalTok{mean )}
\FunctionTok{autoplot}\NormalTok{(Inflow)}
\end{Highlighting}
\end{Shaded}

\includegraphics{TSA_A7_Sp21_files/figure-latex/unnamed-chunk-10-1.pdf}

\hypertarget{q10}{%
\subsubsection{Q10}\label{q10}}

From the plot in Q9 which model or model(s) are leading to the better
forecasts? Explain your answer. Hint: Think about which models are doing
a better job forecasting the high and low inflow months for example.

\begin{quote}
Answer: It seems like the State Space model has the most accurate
forecast, corresponding to the original data.
\end{quote}

\hypertarget{q11}{%
\subsubsection{Q11}\label{q11}}

Now compute the following forecast metrics we learned in class: RMSE and
MAPE, for all the models you plotted in part Q9. You can do this by hand
since you have forecasted and observed values for the year of 2010. Or
you can use R function \(accuracy()\) from package ``forecast'' to do
it. Build and a table with the results and highlight the model with the
lowest MAPE. Does the lowest MAPE corresponds match your answer for part
Q10?

\begin{Shaded}
\begin{Highlighting}[]
\NormalTok{last\_obs }\OtherTok{\textless{}{-}}\NormalTok{ lastyear\_ts[(}\DecValTok{132{-}11}\NormalTok{)}\SpecialCharTok{:}\DecValTok{132}\NormalTok{]}

\NormalTok{ARIMA\_scores }\OtherTok{\textless{}{-}} \FunctionTok{accuracy}\NormalTok{(Temp3}\SpecialCharTok{$}\NormalTok{mean, last\_obs)}
\NormalTok{SARIMA\_scores }\OtherTok{\textless{}{-}} \FunctionTok{accuracy}\NormalTok{(SARIMA\_forecast}\SpecialCharTok{$}\NormalTok{mean, last\_obs)}
\NormalTok{Exp\_scores }\OtherTok{\textless{}{-}} \FunctionTok{accuracy}\NormalTok{(Exp\_smooth}\SpecialCharTok{$}\NormalTok{forecast, last\_obs)}
\NormalTok{SS\_scores }\OtherTok{\textless{}{-}} \FunctionTok{accuracy}\NormalTok{(SS\_for}\SpecialCharTok{$}\NormalTok{mean, last\_obs)}

\NormalTok{seas\_scores }\OtherTok{\textless{}{-}} \FunctionTok{as.data.frame}\NormalTok{(}\FunctionTok{rbind}\NormalTok{(ARIMA\_scores, SARIMA\_scores, Exp\_scores, SS\_scores))}
\FunctionTok{row.names}\NormalTok{(seas\_scores) }\OtherTok{\textless{}{-}} \FunctionTok{c}\NormalTok{(}\StringTok{"ARIMA"}\NormalTok{, }\StringTok{"SARIMA"}\NormalTok{, }\StringTok{"Exp"}\NormalTok{, }\StringTok{"SS"}\NormalTok{)}

\CommentTok{\#choose model with lowest RMSE}
\NormalTok{best\_model\_index }\OtherTok{\textless{}{-}} \FunctionTok{which.min}\NormalTok{(seas\_scores[,}\StringTok{"RMSE"}\NormalTok{])}
\FunctionTok{cat}\NormalTok{(}\StringTok{"The best model by RMSE is:"}\NormalTok{, }\FunctionTok{row.names}\NormalTok{(seas\_scores[best\_model\_index,]))}
\end{Highlighting}
\end{Shaded}

\begin{verbatim}
## The best model by RMSE is: SS
\end{verbatim}

\begin{Shaded}
\begin{Highlighting}[]
\FunctionTok{kbl}\NormalTok{(seas\_scores, }\AttributeTok{caption =} \StringTok{"Forecast Accuracy for Seasonal Data"}\NormalTok{, }\AttributeTok{digits =} \FunctionTok{array}\NormalTok{(}\DecValTok{4}\NormalTok{,}\FunctionTok{ncol}\NormalTok{(seas\_scores))) }\SpecialCharTok{\%\textgreater{}\%} 
  \FunctionTok{kable\_styling}\NormalTok{(}\AttributeTok{full\_width =} \ConstantTok{FALSE}\NormalTok{, }\AttributeTok{position =} \StringTok{"center"}\NormalTok{) }\SpecialCharTok{\%\textgreater{}\%}
  \FunctionTok{kable\_styling}\NormalTok{(}\AttributeTok{latex\_options=}\StringTok{"striped"}\NormalTok{, }\AttributeTok{stripe\_index =} \FunctionTok{which.min}\NormalTok{(seas\_scores[,}\StringTok{"RMSE"}\NormalTok{]))}
\end{Highlighting}
\end{Shaded}

\begin{table}

\caption{\label{tab:unnamed-chunk-11}Forecast Accuracy for Seasonal Data}
\centering
\begin{tabular}[t]{l|r|r|r|r|r}
\hline
  & ME & RMSE & MAE & MPE & MAPE\\
\hline
ARIMA & -5818.4498 & 5852.520 & 5818.450 & -38.0417 & 38.0417\\
\hline
SARIMA & -4031.8177 & 4171.259 & 4031.818 & -23.8732 & 23.8732\\
\hline
Exp & -6502.7059 & 8098.586 & 6502.706 & -26.2114 & 26.2114\\
\hline
\cellcolor{gray!6}{SS} & \cellcolor{gray!6}{-106.3944} & \cellcolor{gray!6}{1332.997} & \cellcolor{gray!6}{1076.021} & \cellcolor{gray!6}{0.6930} & \cellcolor{gray!6}{5.6184}\\
\hline
\end{tabular}
\end{table}

\begin{quote}
Answer: Yes, it seems like the lowest MAPE corresponds to my answer in
Q10.
\end{quote}

\end{document}
